% !TEX TS-program = pdflatex
% !TEX encoding = UTF-8 Unicode

% This file is a template using the "beamer" package to create slides for a talk or presentation
% - Giving a talk on some subject.
% - The talk is between 15min and 45min long.
% - Style is ornate.

% MODIFIED by Jonathan Kew, 2008-07-06
% The header comments and encoding in this file were modified for inclusion with TeXworks.
% The content is otherwise unchanged from the original distributed with the beamer package.

\documentclass{beamer}


% Copyright 2004 by Till Tantau <tantau@users.sourceforge.net>.
%
% In principle, this file can be redistributed and/or modified under
% the terms of the GNU Public License, version 2.
%
% However, this file is supposed to be a template to be modified
% for your own needs. For this reason, if you use this file as a
% template and not specifically distribute it as part of a another
% package/program, I grant the extra permission to freely copy and
% modify this file as you see fit and even to delete this copyright
% notice. 


\mode<presentation>
{
  \usetheme{Warsaw}
  % or ...

  \setbeamercovered{transparent}
  % or whatever (possibly just delete it)
}


\usepackage[english]{babel}
% or whatever

\usepackage[utf8]{inputenc}
% or whatever

\usepackage{times}
\usepackage[T1]{fontenc}
% Or whatever. Note that the encoding and the font should match. If T1
% does not look nice, try deleting the line with the fontenc.

\usepackage{alltt}
\usepackage{amsmath}

\title % (optional, use only with long paper titles)
{Reasoning with Prioritized Defaults}

%TODO??
%\subtitle
%{Presentation Subtitle} % (optional)

\author[Michail Gelfond, Tran Cao Son] % (optional, use only with lots of authors)
{Michail Gelfond, Tran Cao Son}
% - Use the \inst{?} command only if the authors have different
%   affiliation.

\institute[Universities of Somewhere and Elsewhere] % (optional, but mostly needed)
{
  Computer Science Department\\
  University of Texas at El Paso
}
% - Use the \inst command only if there are several affiliations.
% - Keep it simple, no one is interested in your street address.

\date[Short Occasion] % (optional)
{1998}

\subject{Talks}
% This is only inserted into the PDF information catalog. Can be left
% out. 

%TODO

% If you have a file called "university-logo-filename.xxx", where xxx
% is a graphic format that can be processed by latex or pdflatex,
% resp., then you can add a logo as follows:

% \pgfdeclareimage[height=0.5cm]{university-logo}{university-logo-filename}
% \logo{\pgfuseimage{university-logo}}



% Delete this, if you do not want the table of contents to pop up at
% the beginning of each subsection:
\AtBeginSubsection[]
{
  \begin{frame}<beamer>{Outline}
    \tableofcontents[currentsection,currentsubsection]
  \end{frame}
}


% If you wish to uncover everything in a step-wise fashion, uncomment
% the following command: 

%\beamerdefaultoverlayspecification{<+->}


\begin{document}

\begin{frame}
  \titlepage

\begin{center}
\vspace{-30pt}
Presented by: \\
Jakub Podlaha, \\
Vladislav Vavra
\end{center}

\end{frame}

\begin{frame}{Outline}
  \tableofcontents
  % You might wish to add the option [pausesections]
\end{frame}


% Since this a solution template for a generic talk, very little can
% be said about how it should be structured. However, the talk length
% of between 15min and 45min and the theme suggest that you stick to
% the following rules:  

% - Exactly two or three sections (other than the summary).
% - At *most* three subsections per section.
% - Talk about 30s to 2min per frame. So there should be between about
%   15 and 30 frames, all told.

%%%%%%%%%%%%%%%%%%%%%%%%%%%%%%%%%%%%%%%%%%%%%%%%%%%%%%%%%%%%%%%%%%%%%%%%%%%%%%%%%%%%%%%%

\section{Introduction}

  \subsection[Motivation]{Why Prioritize Defaults?}

  \begin{frame}{Why Prioritized Defaults?} %{Subtitles are optional.}
    % - A title should summarize the slide in an understandable fashion
    %   for anyone how does not follow everything on the slide itself.
    \begin{itemize}
      \item Defaults in natural language
      \item Defaults with contradictory conclusions
      \item Expressing relative strengths of defaults
      \item e.g. Legal Reasoning, Reasoning with Experts Knowledge
    \end{itemize}
  \end{frame}
  
  \subsection[Approaches]{Two Possible Approaches}

  \begin{frame}{Two Possible Approaches}
    \begin{itemize}
      \item Develop language to express prioritized defaults (special~semantics)
      \begin{itemize}
        \item ex. generate answer sets and reduce by preference relation
      \end{itemize}
      \item Use standard Logic programming augmented by the~preference relation
    \end{itemize}
  \end{frame}

  \subsection[The Paper's Approach]{The Paper's Approach}

  \begin{frame}{The Paper's Approach}
    \begin{itemize}
      %\item Design simple language %$\mathcal{L}$
      \item Understand narrow sence "a's are normally b's"
      \item Allow dynamic priorities (i.e. defaults \\ with preference relation)
      \item Elaboration tolerant
      \item Give semantics without new general purpose nonmonotonic formalism
      \item Some inference mechanism already available
    \end{itemize}
  \end{frame}


\section{The Language of Prioritized Defaults}

  \subsection{Hello Priorities - Language Demo}
  
  \begin{frame}{Example - Programming Students}
    \begin{alltt}
student(mary). dept(cs). \ is\_in(mary,cs). \\
student(mike). dept(art). is\_in(mike,art).\\
student(sam). \ dept(cis). is\_in(sam,cis). \\ ~ \\
default(d1(S), -can\_progr(S), [student(S)]). \\
default(d2(S), \ can\_progr(S), [student(S), \\
\ \ \ \ \ \ \ \ \ \ \ \ \ \ \ \ \ \ \ \ \ \ \ \ \ \ \ \ \ \ \ is\_in(S,cs)]).\\ ~ \\
prefer(d2(S), d1(S)). \\ ~ \\
rule(r1(S), can\_read(S), [student(S)]).
    \end{alltt}
  \end{frame}

  \subsection{Domain Description}
  
  \begin{frame}{Domain Description}
    %Logic program composed of: %$\mathcal{P}$
    %\begin{itemize}
      %\item Domain description %(denoted $\mathcal{D}$)
      \begin{itemize}
        \item Facts 
        \begin{itemize}
          \item all the facts from our original domain
          \item \texttt{prefer(D1, D2)} 
          \item \texttt{conflict(D1, D2)}
        \end{itemize}
        \item Laws
        \begin{itemize}
          %\item i.e. statements in form default(D, L, Body) and \mbox{prefer(D1, D2).} 
          \item \texttt{rule(R, H, Body)} 
          \item \texttt{default(D, H, Body)}
        \end{itemize}
      \end{itemize}
    %  \item Domain independent axioms %(denoted $\mathcal{P}_\sigma$)
      %\item notion of entailment between query and domain description
    %\end{itemize}
  \end{frame}

  \begin{frame}{Domain Description - Refinement of Facts}

    Basic relations

    \begin{itemize}
      \item \texttt{holds(l)} denotes that literal \texttt{l} strictly holds  
      \item \texttt{holds\_by\_default(l)}
       denotes that literal \texttt{l} holds defeasibly i.e. by some default
    \end{itemize}

    Translating facts of our domain into ASP

    \begin{itemize}
      \item $\mathcal{P}(\mathcal{D}) = \mathcal{P} \cup \{ holds(l) \ | \ l \in fact(\mathcal{D}) \} \cup laws(\mathcal{D})$. 
    \end{itemize}

  \end{frame}

  %\begin{frame}{Entailment in Domain Description}
  %  \textbf{Definition 2.2}
  %  We say that a domain description $\mathcal{D}$ entails
  %  a~query q $(\mathcal{D} \models q)$ if q belongs to every answer set
  %  of the~program \\
  %  $\mathcal{P}_\sigma(\mathcal{D}) = \mathcal{P}_\sigma \cup \{ holds(l) \ | \ l \in fact(\mathcal{D}) \} \cup laws(\mathcal{D})$. 
  %
  %  $laws(\mathcal{D})$ denotes set of statements of the form \ref{item:rule1} and \ref{item:rule2}
  %\end{frame}

  %\begin{frame}{Terms to Describe Defaults and Strict Rules}
  %  \begin{alignat}{1}
  %  & \hspace*{-5cm} \label{item:rule1} rule(r_0, l_0, [l_1, \dots l_m]) \\
  %  & \hspace*{-5cm} \label{item:rule2} default(d_0, l_0, [l_1, \dots l_m]) \\
  %  & \hspace*{-5cm} \label{item:rule3} conflict(d_1, d_2) \\
  %  & \hspace*{-5cm} \label{item:rule4} prefer(d_1, d_2)
  %  \end{alignat}
  %\end{frame}
  
  %\begin{frame}{Domain Description - Intuition}
  %  \begin{itemize}
  %    \item "logic counter-part" as intuitive explanaiton\\
  %    \item rule $r_0$:    $ l_0 \leftarrow l_1, \dots, l_n. $
  %    \item default $d_0$: $ l_0 \leftarrow l_1, \dots, l_n, not ~ \neg l_0. $
  %  \end{itemize}
  %\end{frame}
  
  \subsection{Domain Independent Axioms}
  
  \begin{frame}{Axioms}
    \begin{itemize}
      \item handout + slides
    \end{itemize}
  \end{frame}
  
  
  \begin{frame}{Examples}
    \begin{itemize}
      \item go through programming students example, show it running on computer
      \item example of flying penguin, ask them and let them quess,
        then show the correct answer on pc
      \item maybe use example 3.3 (not yet ready)
    \end{itemize}
  \end{frame}
  
  
  \begin{frame}{Extending the language}
    \begin{itemize}
      \item $default(d,l_0,[l_1, l_2, ... l_m], [l_m+1, ... l_n])$
      \item show new axioms
      \item example 4.1 : who can vote?
    \end{itemize}
  \end{frame}
  
  
  \begin{frame}{Weak Exceptions}
    \begin{itemize}
      \item "do not apply default d to objects satisfiyng property p" 
      \item How to do it?
      \item $exception(d, [l_1, ... ,l_n] [l_n+1,...ln+m])$
      \item Add new axiom: defeated(D) :- exception ......    
    \end{itemize}
  \end{frame}
  
  
  \begin{frame}{Cautions reasoning}
    \begin{itemize}
      \item Two contrary defaults - no answer set will be resolved
      \item Add new axiom
    \end{itemize}
  \end{frame}
  
  
  
  
  
\section*{Summary}

\begin{frame}{Summary}
%
%  % Keep the summary *very short*.
%  \begin{itemize}
%  \item
%    The \alert{first main message} of your talk in one or two lines.
%  \item
%    The \alert{second main message} of your talk in one or two lines.
%  \item
%    Perhaps a \alert{third message}, but not more than that.
%  \end{itemize}
%  
%  % The following outlook is optional.
%  \vskip0pt plus.5fill
%  \begin{itemize}
%  \item
%    Outlook
%    \begin{itemize}
%    \item
%      Something you haven't solved.
%    \item
%      Something else you haven't solved.
%    \end{itemize}
%  \end{itemize}
\end{frame}


\end{document}


